\usepackage[resetfonts]{cmap}
\usepackage[english]{babel}
\usepackage[T1,T2A]{fontenc}
\usepackage{amsmath}
\usepackage{amsthm}
\usepackage{amsfonts}
\usepackage{markdown}
\usepackage{listings}
\lstset{
        mathescape,
        numbers=left
}
\usepackage{enumitem}
%\usepackage[xindy]{imakeidx}
\usepackage[backend=biber,
            style=ieee,
            maxnames=6,
            alldates=iso8601]{biblatex}
\usepackage{hyperref}
\usepackage{url}
\addbibresource{literature.bib}
\usepackage{tikz}
\usetikzlibrary{shapes, arrows, positioning, calc, fit, backgrounds,
  decorations.pathmorphing, decorations.markings, shadows}
\newcommand{\sbt}{Symbiotic}
\thesissetup{
  date = \the\year/\the\month/\the\day,
  university = mu,
  faculty = fi,
  type = bc,
  author = Tomáš Jašek,
  title = {Improvements of reaching definitions analysis in Symbiotic},
  TeXtitle = {Improvements of reaching definitions analysis in Symbiotic},
  advisor = {Mgr. Marek Chalupa\\ \textbf{Consultant}: doc. RNDr. Jan Strejček, Ph.D.},
  keywords = {C++, formal methods, Formela, implementation, static analysis, Symbiotic },
  TeXkeywords = {C++, formal methods, Formela, implementation, static analysis, Symbiotic },
  abstract = {The aim of this thesis is to improve reaching
  definitions analysis that is part of software verification tool
  \sbt. The original ``dense'' analysis propagates all reaching
  definitions throughout the whole control flow graph of a program.
  The new, improved analysis is based on transformation of memory
  operations in a program to static single assignment form, which
  directly reveals reaching definitions.  With the new analysis,
  \sbt{} is faster and uses less memory than before. },
  thanks = {I would like to thank to my advisor and consultant for their time
  and advice they provided during my work on this thesis. This
  thesis would not have been possible without them.

I would also like to say thank you to my family, friends and everyone who
    supported me during my studies.
    }
}

\usepackage{algorithm2e}
\DontPrintSemicolon
\SetKwProg{Fn}{function}{}{}
\SetKw{Nil}{nil}
\SetKw{Return}{return}
\addtocontents{toc}{\setcounter{tocdepth}{1}}